\documentclass[a4paper, 11pt]{article}

\begin{document}

\section{Introduction}

This document describes the implementation of the CCP4 Scaler - this uses 
standard CCP4 programs to present the xia2dpa Scaler interface, including:

\begin{itemize}
\item{\textsc{Pointless}}
\item{\textsc{Cad}, \textsc{Sortmtz}, \textsc{Reindex}, \textsc{Mtzdump}
\textsc{Mtz2various}}
\item{\textsc{Scala}}
\end{itemize}

\noindent
This number of programs is necessary to provide the full functionality,
including implicit reindexing to the correct setting, implicit determination
of pointgroup, implicit combination and separation of reflection files.

This document describes how this module works and what is going on.

\section{Detecting Radiation Damage A: Within A Sweep}

Use Scala - look at $R_{\rm{merge}}$ vs. batch. Don't forget that there 
are cases which Ed has given me where this behaves strangely - also need
a benchmark for this (e.g. TS03 INFL, LREM curves.)

\section{Detecting Radiation Damage B: Between Sweeps}

Use Scaleit - look at the relative $B$ factor between data sets.

\section{Deciding Resolution Limits}

Should the resolution limits be decided as one limit for all data sets or
a number of limits? Need to decide what I am going to do with the data 
afterwards (in particular, how does solve/resolve cope?)



\section{Setting Error Parameters}

In integration the errors are usually underestimated (Evans, sometime.) 
Scala includes standard deviation inflation parameters SdAdd, SdFac and SdB
for full and partial reflections for each run - there are therefore, in 
principle, 6 parameters for each run to be adjusted. However, the SdFac
(an overall scaling) is automatically determined by Scala, leaving the 
other four. SdB, which corrects the overall curvature, is usually a number
around 10, while the SdAdd of typically 0.02 (the default) affects the 
overall shape and gradient of the curve.

In the summary the standard deviations for each integration bin should 
be $\sim 1$.

\end{document}
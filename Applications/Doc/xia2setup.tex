\documentclass[a4paper, 11pt]{article}
\begin{document}
\section{Application: xia2setup}
\subsection{Introduction}

xia2setup is a new application which will ``trawl'' a directory and construct
a .xinfo file to reduce the data that is found, with the following assumptions:

\begin{itemize}
\item{All data in a directory is measured from one crystal.}
\item{The date and time of data collection is in the image header.}
\item{If a sequence (\verb|.seq|) file is found in the directory this
corresponds to the sequence of that crystal.}
\item{If a scan (\verb|.scan|) file is found this corresponds to the
wavelengths of the data which were collected.}
\item{Heavy atom information will be passed in on the command line.}
\end{itemize}

This will give a command line something like:

\begin{verbatim}
xia2setup -atom se /path/to/directory
\end{verbatim}

This will generate a \verb|.xinfo| file called \verb|automatic.xinfo|
which should be inspected and perhaps updated before being used. If heavy
atom information and a scan are supplied then the system will try to identify 
PEAK, INFLECTION, LOW and HIGH REMOTE data sets, or SAD if appropriate, else 
for sigle wavelength sets this will be called ``NATIVE'' or WAVE1 ...

\end{document}


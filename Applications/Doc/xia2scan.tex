\documentclass[a4paper, 11pt]{article}
\begin{document}

\section{Introduction}

The application \verb|xia2scan| is designed to scan through a list of 
provided images and print information about the ``quality'' of the 
diffraction on the images. This will end up as an executable
\verb|xia2-scan(=> .sh/.bat)| and a python program xia2scan.py.

\subsection{Appication}

This is designed to be useful for 

\begin{itemize}
\item{Selecting the best images for autoindexing from a sweep.}
\item{Optimizing humidity when a humidifier is available.}
\end{itemize}

\noindent
by printing information about the strength of diffraction in each image.

\section{Uses...}

This uses the wrappers for labelit.screen and labelit.stats\_distl.

\section{Dependencies}

This application will depend on having access to a user provided beam centre
(by implication then an input argument / command line handler) and also 
diffraction image name parsing to allow searching of a directory for
matching images.

\section{Use Cases}

\subsection{UC 1: Humidifier}

Requires:

\begin{itemize}
\item{The correct beam position (for indexing from single images.)}
\item{The images.}
\end{itemize}

Provides:

\begin{itemize}
\item{The statistics for each image in the list.}
\end{itemize}

\subsection{UC 2: Indexing Image Selection}

This will scan through all of the images in a sweep and select, based on the
spot separation and so on, the best pair for autoindexing.

\section{Implementation}

\subsection{UC 1: Humidifier}

This will use the following classes:

\begin{itemize}
\item{Schema.Sweep.SweepFactory - to generate a list of sweeps from the 
images provided.}
\item{Handlers.CommandLine - to hold the beam centre information from a
previous indexing run.} 
\end{itemize}

To achieve reasonable speed, the labelit run will not optimise the beam
centre (this is unreliable from a single image anyway) and also will 
run the labelit before distl, and use labelit.stats\_distl to get the results
afterwards.

This should do all of the running, then collate the results and print a
summary. The following arc should be followed:

\begin{itemize}
\item{[determine accurate beam centre]}
\item{Digest matching frames to list of sweeps.}
\item{For each sweep, run labelit, get an idea of the unit cell volume,
run labelit.stats\_distl to get the intensity summary, save.}
\item{Print summary by image number.}
\end{itemize}

\subsubsection{Test Data}

Rajan provided me with a couple of test data sets - use these. Oh bugger - the 
images are stills... Add a special case in the sweep factory to return these 
as separate frames? Or just plough ahead??

\end{document}
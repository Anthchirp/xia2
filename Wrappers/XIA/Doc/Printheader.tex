\documentclass[a4paper, 11pt]{article}
\begin{document}

\section{Introduction}

This document describes the Printheader class, which wraps the program 
printheader to provide information about the image headers. This will
be used in the Dataset object.

\section{Input}

All this takes as input is a frame set in setImage().

\section{Output}

The following results are produced:

{
\small
\begin{verbatim}
{'exposure_time': 5.0, 
 'distance': 170.0, 
 'phi_start': 290.0, 
 'phi_end': 291.0, 
 'epoch': 1096203695.0, 
 'date': 'Sun Sep 26 14:01:35 2004', 
 'wavelength': 0.979660, 
 'detector': 'adsc'}
\end{verbatim}
}

\section{Updates}

\subsection{15/JUN/06}

Added a HeaderCache singleton to record the image headers as they are
read, so that when they are asked for again the results come back much 
more rapidly.

\subsection{13/JUL/06}

This is too slow - there is also something wrong with the application xia2find,
since this takes far too long...

Actually I am not sure that this is the best way to go - I suspect that I need
to write this in Python to access the images directly.

\end{document}


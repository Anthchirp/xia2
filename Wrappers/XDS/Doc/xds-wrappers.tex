\documentclass[a4paper, 11pt]{article}
\begin{document}
\section{Implementation of Data Processing with 
XDS}

\subsection{Introduction}

This document describes the implementation of the xia2 data processing 
interfaces (indexing, integration and scaling) with XDS. There are
two possibilities for this:

\begin{itemize}
\item{Processing with only XDS, XSCALE - that is, not assuming that pointless,
scala etc can be used.}
\item{Processing with XDS, XSCALE and CCP4 tools - this will probably be more
like the CCP4 Scaler implementation.}
\end{itemize}

Perhaps there should be two implementations...?

\subsection{Indexing}

\begin{verbatim}
xycorr -> init -> colspot -> idxref -> [select] -> idxref
\end{verbatim}

This will list all possible indexing solutions - the highest will be selected
for integration as per the current implementation with Mosflm.

\subsection{Integration 1}

\begin{verbatim}
defpix -> integrate [-> correct]
\end{verbatim}

This may recycle if the integration fails... for instance, I need to see how
the TS01 NATIVE is handled by XDS integrating in oI, when mC is correct.

\subsection{Pointgroup Determination}

\begin{verbatim}
integrate.hkl -> combat -> pointless -> [select indexing solution]
\end{verbatim}

This once again needs to be linked in to the indexer as per the 
CCP4 Scaler implementation. 

\subsection{Integration 2}

\begin{verbatim}
[select indexing solution] -> defpix -> integrate -> correct [in correct pg]
if fail, --spacegroup; repeat
\end{verbatim}

This may work out as being the same module as Integration 1, if it is
implemented in the same way as the CCP4 system.

\subsection{Analysing Integration Output}

For the 1VR9/TS01 NATIVE data set the indexing solution gives oI as the 
correct lattice. In integration, the STANDARD DEVIATION IN SPOT POSITION 
starts off rather high, then drops a little, then rises significantly again.
In addition, the spot profiles in the integration output look rather smeared.
This means that the output of both need to be analysed. This analysis should 
look at the centres of mass of the spots and also the second moments - 
that is the elipsoid which defines this, and assess how ``spherical''
this is.

Uniformity is the key, because we don't know the absolutely correct values.
For the standard deviation, if it varies by more than say a factor of 1.5 or 
2 then a problem should be raised.

\subsection{Reindexing \& Spacegroup Determination}

\begin{verbatim}
[sweep 1] -> [quick scale]
[n] * xds_ascii.hkl -> pointless [reindex against quick scale] -> correct
\end{verbatim}

\subsection{Scaling \& Merging}

\begin{verbatim}
xscale [output unmerged] -> scala
\end{verbatim}

\end{document}

